% \iffalse
\let\negmedspace\undefined
\let\negthickspace\undefined
\documentclass[journal,12pt,twocolumn]{IEEEtran}
\usepackage{cite}
\usepackage{amsmath,amssymb,amsfonts,amsthm}
\usepackage{algorithmic}
\usepackage{graphicx}
\usepackage{textcomp}
\usepackage{xcolor}
\usepackage{txfonts}
\usepackage{listings}
\usepackage{enumitem}
\usepackage{mathtools}
\usepackage{gensymb}
\usepackage{comment}
\usepackage[breaklinks=true]{hyperref}
\usepackage{tkz-euclide} 
\usepackage{listings}
\usepackage{gvv}                                        
\def\inputGnumericTable{}                                 
\usepackage[latin1]{inputenc}                                
\usepackage{color}                                            
\usepackage{array}                                            
\usepackage{longtable}                                       
\usepackage{calc}                                             
\usepackage{multirow}                                         
\usepackage{hhline}                                           
\usepackage{ifthen}                                           
\usepackage{lscape}

\newtheorem{theorem}{Theorem}[section]
\newtheorem{problem}{Problem}
\newtheorem{proposition}{Proposition}[section]
\newtheorem{lemma}{Lemma}[section]
\newtheorem{corollary}[theorem]{Corollary}
\newtheorem{example}{Example}[section]
\newtheorem{definition}[problem]{Definition}
\newcommand{\BEQA}{\begin{eqnarray}}
\newcommand{\EEQA}{\end{eqnarray}}
\newcommand{\define}{\stackrel{\triangle}{=}}
\theoremstyle{remark}
\newtheorem{rem}{Remark}
\begin{document}
\bibliographystyle{IEEEtran}
\vspace{3cm}

\title{NCERT-discrete : 11.9.3 - 21}
\author{EE23BTECH11025 - Anantha Krishnan $^{}$% <-this % stops a space
}
\maketitle
\newpage
\bigskip

\renewcommand{\thefigure}{\theenumi}
\renewcommand{\thetable}{\theenumi}
\section{question}
Find four numbers forming a geometric progression in which the third term is greater than the first term by 9, and the second term is greater than the $4^{th}$ by 18.\\

\textbf{Solution}:
Let's assume the 4 terms of the geometric sequences are a, ar, a{r^2}, a{r^3}.
\\

Here,the common ratio is r.
It is given that

\begin{align}
a{r^2}-a=9\label{eq:1} \\
ar-a{r^3}=18\label{eq:2}
\end{align}
Now, we solve $\eqref{eq:1}$ and $\eqref{eq:2}$\\
From $\eqref{eq:1}$
\begin{align}
         a(r^2-1)=9
         \end{align}

     
\\Therefore 
\begin{align}
         (r^2-1)=\dfrac{9}{a}
         \end{align}

     \\also from $\eqref{eq:2}$
     \begin{align}
     ar(1-r^2)=18
     \end{align}
     
     \\Putting the value of ${r^2}-1$ in the above equation gives 
     \begin{align}
              ar(\dfrac{-9}{a})=18
                   \end{align}

     Which in solving gives r=-2. Putting the value of r in $\eqref{eq:1}$ yields a=3.
     Therefore , the final geometric series is 3 , -6 , 12 , -24.





 






\end{document}
