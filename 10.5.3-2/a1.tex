% \iffalse
\let\negmedspace\undefined
\let\negthickspace\undefined
\documentclass[journal,12pt,twocolumn]{IEEEtran}
\usepackage{cite}
\usepackage{amsmath,amssymb,amsfonts,amsthm}
\usepackage{algorithmic}
\usepackage{graphicx}
\usepackage{textcomp}
\usepackage{xcolor}
\usepackage{txfonts}
\usepackage{listings}
\usepackage{enumitem}
\usepackage{mathtools}
\usepackage{gensymb}
\usepackage{comment}
\usepackage[breaklinks=true]{hyperref}
\usepackage{tkz-euclide} 
\usepackage{listings}
\usepackage{gvv}                                        
\def\inputGnumericTable{}                                 
\usepackage[latin1]{inputenc}                                
\usepackage{color}                                            
\usepackage{array}                                            
\usepackage{longtable}                                       
\usepackage{calc}                                             
\usepackage{multirow}                                         
\usepackage{hhline}                                           
\usepackage{ifthen}                                           
\usepackage{lscape}

\newtheorem{theorem}{Theorem}[section]
\newtheorem{problem}{Problem}
\newtheorem{proposition}{Proposition}[section]
\newtheorem{lemma}{Lemma}[section]
\newtheorem{corollary}[theorem]{Corollary}
\newtheorem{example}{Example}[section]
\newtheorem{definition}[problem]{Definition}
\newcommand{\BEQA}{\begin{eqnarray}}
\newcommand{\EEQA}{\end{eqnarray}}
\newcommand{\define}{\stackrel{\triangle}{=}}
\theoremstyle{remark}
\newtheorem{rem}{Remark}
\begin{document}
\bibliographystyle{IEEEtran}
\vspace{3cm}

\title{NCERT-discrete : 10.5.3 - 2}
\author{EE23BTECH11025 - Anantha Krishnan $^{}$% <-this % stops a space
}
\maketitle
\newpage
\bigskip

\renewcommand{\thefigure}{\theenumi}
\renewcommand{\thetable}{\theenumi}

\section{question}
\vspace{0.5cm}
Find the sums given below:
\begin{enumerate}
    \item[(i)] 7 + $10\dfrac{1}{2}$ + 14 ... + 84
    \item[(ii)] 34 + 32 + 30 ... + 10

    \item[(iii)] -5 + -8 + -11 ... -230

\end{enumerate}


\vspace{0.5cm}
\textbf{Solutions}:
\begin{enumerate}
\item[(i)]   

By observing the consecutive common differences in the given series , we observe that it is a constant value , which is $\dfrac{7}{2}$ .\\
Since this an arithmetic progression , we can use the formula which dictates the sum of "n" terms of such a series

Let "$S_n$" denote the sum of n terms in a series ,"a" denotes its first term and "d" denotes the common difference.It is known that
\begin{align}
{S_n} = \dfrac{n}{2}(2a + (n-1)d)\label{eq:1}
\end{align}
In the question, a=7 and d=$\dfrac{7}{2}$ , and "n" is unknown\\
For calculating the number of terms, we use the formula
\begin{align}
{T_n} = a + (n-1)d\label{eq:2}
\end{align}
Where $T_n$ is the nth term of the series\\
Given that $T_n$ is 84, we solve for "n"
\begin{align}  
84 = 7+(n-1)\dfrac{7}{2}
\end{align}
Solving this yields n=23.\\
We now use this result for calculating $S_{23}$
\begin{align}
    S_{23} = \dfrac{23}{2}(14+(22)\dfrac{7}{2})
    \end{align}
Again, solving this yields $S_{23}$ as 1046.5\\\\

\item[(ii)]

\vspace{0.5cm}
Based on the analysis of the previous bit, we observe that in this bit \\ a=34 , d=-2
For calculating the number of terms, we use the formula (\eqref{eq:2}\\
Substituting the values, we get
\begin{align}
     10 = 34 + (n-1)(-2)
     \end{align}
Solving this yields n=13\\
For calculating the sum , we use \eqref{eq:1}
\begin{align}
 S_{13} = \dfrac{13}{2}(64+11(-2))
 \end{align}

 Solving this, we get $S_n$= 286.\\\\

\item[(iii)]  
\vspace{0.5cm}
By using the previous analysis, we can conclude that a=-5, d=-3\\
Again, for n , we use the formula \eqref{eq:2}
\begin{align}
-230= -5+(n-1)(-3)
\end{align}
Solving this yields n=76\\
Now, for the sum we use equation \eqref{eq:1} :\\
\begin{align}
    S_{76}=\dfrac{76}{2}(-10+(76-1)(-3))
    \end{align}
Solving this we obtain $S_{76}$=-8930.\\\\
\textbf{Extension}:
It is told to replace $S_n$ with x(n) and therefore calculate X(Z). The equation $\eqref{eq:1}$ and calculate the general form of X(Z) .
It is known that\\
\begin{align}
 \sum_{n=-\infty}^{\infty} Z^{-n}x(n) = X(Z)\label{eq:3}
 \end{align}
For n$\epsilon$[-\infty,0] 

The summation is 0 in the above question. So we can start the sum from n=1 and so on...
Putting the equation $\eqref{eq:1}$ in $\eqref{eq:3}$\\
\begin{align}
    \sum_{n=1}^{\infty} Z^{-n}\dfrac{n}{2}(2a + (n-1)d) = X(Z) \label{eq:4}
\end{align}
Writing another equation by multiplying $\eqref{eq:4}$ with Z, we get
\begin{align}    
\sum_{n=1}^{\infty} Z^{-n+1}\dfrac{n}{2}(2a + (n-1)d) = ZX(Z)\label{eq:5}
\end{align}
Now we subtract $\eqref{eq:4}$ from $\eqref{eq:5}$ by displacing it with one term , i.e we subtract the first term of $\eqref{eq:4}$ from the second term of $\eqref{eq:5}$ and so on.\\
By simplifying this, we get
\begin{align}
    \sum_{n=0}^{\infty} (a-\dfrac{d}{2})Z^{-n}+\dfrac{n}{2}dZ^{-n} = ZX(Z)-X(Z)
\end{align}
The first part is a GP and the second part is an AGP. Since the nature of Z is not known, we calculate the general sum for finite terms(Assuming k) and then use it for each bit.\\
Calculating the individual sums, we get
\begin{align}
\dfrac{(a-\dfrac{d}{2})Z^{1-k}\dfrac{(Z^{k}-1)}{Z-1}+\dfrac{d}{2Z^{k-2}}\dfrac{Z^{k-1}-1}{2(Z-1)^{2}}-\dfrac{d(k-1)}{2(Z-1)Z^{k-1}}}{Z-1}\\ \notag=X(Z)
\end{align}
Now if we put the values of (i),(ii) and (iii) , we get the respective X(Z) as :\\
 \item \dfrac{(\dfrac{21}{4})Z^{-22}\dfrac{(Z^{23}-1)}{Z-1}+\dfrac{7}{4Z^{21}}\dfrac{Z^{22}-1}{2(Z-1)^{2}}-\dfrac{7(22)}{4(Z-1)Z^{22}}}{Z-1} \\\\
 
 \item \dfrac{35Z^{-12}\dfrac{(Z^{13}-1)}{Z-1}-\dfrac{1}{Z^{11}}\dfrac{Z^{12}-1}{2(Z-1)^{2}}+\dfrac{2(12)}{(Z-1)Z^{12}}}{Z-1} \\\\
 
 \item \dfrac{(\dfrac{7}{2})Z^{-75}\dfrac{(Z^{76}-1)}{Z-1}-\dfrac{-3}{2Z^{74}}\dfrac{Z^{75}-1}{2(Z-1)^{2}}+\dfrac{-3(75)}{2(Z-1)Z^{75}}}{Z-1}
 






\end{enumerate}



 

 






\end{document}
