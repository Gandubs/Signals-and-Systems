% \iffalse
\let\negmedspace\undefined
\let\negthickspace\undefined
\documentclass[journal,12pt,twocolumn]{IEEEtran}
\usepackage{cite}
\usepackage{amsmath,amssymb,amsfonts,amsthm}
\usepackage{algorithmic}
\usepackage{graphicx}
\usepackage{textcomp}
\usepackage{xcolor}
\usepackage{txfonts}
\usepackage{listings}
\usepackage{enumitem}
\usepackage{mathtools}
\usepackage{gensymb}
\usepackage{comment}
\usepackage[breaklinks=true]{hyperref}
\usepackage{tkz-euclide} 
\usepackage{listings}
\usepackage{gvv}                                        
\def\inputGnumericTable{}                                 
\usepackage[latin1]{inputenc}                                
\usepackage{color}                                            
\usepackage{array}                                            
\usepackage{longtable}                                       
\usepackage{calc}                                             
\usepackage{multirow}                                         
\usepackage{hhline}                                           
\usepackage{ifthen}                                           
\usepackage{lscape}

\newtheorem{theorem}{Theorem}[section]
\newtheorem{problem}{Problem}
\newtheorem{proposition}{Proposition}[section]
\newtheorem{lemma}{Lemma}[section]
\newtheorem{corollary}[theorem]{Corollary}
\newtheorem{example}{Example}[section]
\newtheorem{definition}[problem]{Definition}
\newcommand{\BEQA}{\begin{eqnarray}}
\newcommand{\EEQA}{\end{eqnarray}}
\newcommand{\define}{\stackrel{\triangle}{=}}
\theoremstyle{remark}
\newtheorem{rem}{Remark}
\begin{document}
\bibliographystyle{IEEEtran}
\vspace{3cm}

\title{NCERT-discrete : 10.5.3 - 2}
\author{EE23BTECH11025 - Anantha Krishnan $^{}$% <-this % stops a space
}
\maketitle
\newpage
\bigskip

\renewcommand{\thefigure}{\theenumi}
\renewcommand{\thetable}{\theenumi}

\section{question}
A circular disk of mass 10kg is suspended by a wire attached to its centre. The wire is twisted by rotating the disc and released. The period of torsional oscillations is found to be 1.5s. The radius of the disc is 15cm. Determine the torsional spring constant of the wire. (Torsional spring constant $\alpha$ is defined by the relation J=-$\alpha$$\theta$, where J is the restoring couple and $\theta$ is the angle of twist).\\

\textbf{Solution:}
For a torsional pendulum , it is known that the time period(T) of oscillations is given by
\begin{align}
T=2\pi\sqrt{\frac{I}{\alpha}}   \label{eq:1}
\end{align}

where $I$ is the moment of inertia of the system and $\alpha$ is the torsional spring constant.
For calculating the moment of inertia of the bob of the above pendulum about its axis, we use the result
\begin{align}
            I_{disc}=\dfrac{1}{2}mr^2
\end{align}
 where $r$ is the radius of the disc and $m$ is the mass of the pendulum.\\
It is given that m=10kg and r=0.15m
\begin{align}
I_{disc}=\dfrac{1}{2}10(0.15)^2 \:\:kgm^2\\
I_{disc}=0.1125\:\: kgm^2.
\end{align}
Plugging in this value in the equation $\eqref{eq:1}$ Given T=1.5s, we obtain
\begin{align}
    1.5=2\pi\sqrt{\frac{0.1125}{\alpha}}
    \end{align}
Which after solving yields $\alpha$=1.972 $\dfrac{Nm}{rad}$








\end{document}
