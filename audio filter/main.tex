% \iffalse
\let\negmedspace\undefined
\let\negthickspace\undefined
\documentclass[journal,12pt,twocolumn]{IEEEtran}
\usepackage{float}
\usepackage{circuitikz}
\usepackage{cite}
\usepackage{amsmath,amssymb,amsfonts,amsthm}
\usepackage{algorithmic}
\usepackage{graphicx}
\usepackage{textcomp}
\usepackage{xcolor}
\usepackage{txfonts}
\usepackage{listings}
\usepackage{amsmath}
\usepackage{enumitem}
\usepackage{mathtools}
\usepackage{gensymb}
\usepackage{comment}
\usepackage[breaklinks=true]{hyperref}
\usepackage{tkz-euclide} 
\usepackage{listings}
\usepackage{gvv}                                        
\def\inputGnumericTable{}                                 
\usepackage[latin1]{inputenc}                                
\usepackage{color}                                            
\usepackage{array}                                            
\usepackage{longtable}                                       
\usepackage{calc}                                             
\usepackage{multirow}                                         
\usepackage{hhline}                                           
\usepackage{ifthen}                                           
\usepackage{lscape}
\newtheorem{theorem}{Theorem}[section]
\newtheorem{problem}{Problem}
\newtheorem{proposition}{Proposition}[section]
\newtheorem{lemma}{Lemma}[section]
\newtheorem{corollary}[theorem]{Corollary}
\newtheorem{example}{Example}[section]
\newtheorem{definition}[problem]{Definition}
\newcommand{\BEQA}{\begin{eqnarray}}
\newcommand{\EEQA}{\end{eqnarray}}
\newcommand{\define}{\stackrel{\triangle}{=}}
\theoremstyle{remark}
\newtheorem{rem}{Remark}
\begin{document}

\bibliographystyle{IEEEtran}
\vspace{3cm}
\title{Audio Filter Assignment}
\author{EE23BTECH11025 - ANANTHA KRISHNAN$^{*}$% <-this % stops a space
}
\maketitle
\newpage
\bigskip
\renewcommand{\thefigure}{\arabic{figure}}
\renewcommand{\thetable}{\theenumi}

\bibliographystyle{IEEEtran}
\begin{enumerate}[label=\thesection.\arabic*
,ref=\thesection.\theenumi]

\section{Digital Filter}
\label{input}
\item Download the sound file used for this code from the below link
\begin{lstlisting}
link here
\end{lstlisting}
\item 
\label{Python_code_for_filtering}
Listed below is the python code for removal of out of band noise.\lstinputlisting{./codes/audio_filter.py}\label{prob:audio_filter_code}

\item 
\label{Visualization}

The audio file is now analyzed using a spectrogram from the website \href{https://academo.org/demos/spectrum-analyzer}{\url{https://academo.org/demos/spectrum-analyzer}}.\\

 Dark lines in a spectrogram often indicate areas of low intensity and bright lines represent areas of high intensity in the signal. \\
 In the filtered image, there is a higher concentration of bright lines within the cutoff frequency($f_c=1kHz$)


\begin{figure}[H]
    \includegraphics[width=0.8\columnwidth]{figs/Before_audiofilter.png }
    \caption{Spectrogram of the audio file pre Filtering}
    \label{fig:before_filter_plot}
\end{figure}
\begin{figure}[H]
\includegraphics[width=0.8\columnwidth]{figs/After_audiofilter.png}
    \caption{Spectrogram of the audio file post Filtering}
    \label{fig:after_filter_plot}
\end{figure}



\end{enumerate}

\section{Difference Equation}
\begin{enumerate}[label=\thesection.\arabic*,ref=\thesection.\theenumi]
\item Let
\begin{equation}
x(n) = \cbrak{\underset{\uparrow}{1},2,3,4,2,1} \label{prob:2.1}
\end{equation}
Sketch $x(n)$. 
\item Let
\begin{multline}
y(n) + \frac{1}{2}y(n-1) = x(n) + x(n-2), 
\\
y(n) = 0, n < 0 \label{prob:2.2}
\end{multline}
Sketch $y(n)$.\\
Solve\\
\solution  The below C code computes the values of $y(n)$ till $n=20$ 
\begin{lstlisting}
link
\end{lstlisting} 
The following code plots x[n] and $y[n]$
\begin{lstlisting}
link
\end{lstlisting}

\begin{figure}[H]
	\centering
	\includegraphics[width=\columnwidth]{figs/Plot_xn_yn.png}
	\caption{Plot of $x(n)$ and $y(n)$}
	\label{fig:xnyn}
\end{figure}
\end{enumerate}
\section{Z-Transform}

\begin{enumerate}[label=\thesection.\arabic*]
\item The $Z$-transform of $x(n)$ is defined as
%
\begin{equation}
\label{eq:z_trans}
X(z)={\mathcal {Z}}\{x(n)\}=\sum _{n=-\infty }^{\infty }x(n)z^{-n}
\end{equation}
%
Show that
\begin{equation}
\label{eq:shift1}
{\mathcal {Z}}\{x(n-1)\} = z^{-1}X(z)
\end{equation}
and find
\begin{equation}
	{\mathcal {Z}}\{x(n-k)\} 
\end{equation}
\solution From \eqref{eq:z_trans},
\begin{align}
{\mathcal {Z}}\{x(n-k)\} &=\sum _{n=-\infty }^{\infty }x(n-1)z^{-n}
\\
&=\sum _{n=-\infty }^{\infty }x(n)z^{-n-1} = z^{-1}\sum _{n=-\infty }^{\infty }x(n)z^{-n}
\end{align}
resulting in \eqref{eq:shift1}. Similarly, it can be shown that
%
\begin{equation}
\label{eq:z_trans_shift}
	{\mathcal {Z}}\{x(n-k)\} = z^{-k}X(z)
\end{equation}

\item Find
%
\begin{equation}
H(z) = \frac{Y(z)}{X(z)}
\end{equation}
from  \eqref{prob:2.2} assuming that the $Z$-transform is a linear operation.
\\
\solution  Applying \eqref{eq:z_trans_shift} in \eqref{prob:2.2},
\begin{align}
Y(z) + \frac{1}{2}z^{-1}Y(z) &= X(z)+z^{-2}X(z)
\\
\implies \frac{Y(z)}{X(z)} &= \frac{1 + z^{-2}}{1 + \frac{1}{2}z^{-1}}
\label{eq:freq_resp}
\end{align}
%
\item Find the Z transform of 
\begin{equation}
\delta(n)
=
\begin{cases}
1 & n = 0
\\
0 & \text{otherwise}
\end{cases}
\end{equation}
and show that the $Z$-transform of
\begin{equation}
\label{eq:unit_step}
u(n)
=
\begin{cases}
1 & n \ge 0
\\
0 & \text{otherwise}
\end{cases}
\end{equation}
is
\begin{equation}
U(z) = \frac{1}{1-z^{-1}}, \quad \abs{z} > 1
\end{equation}
\solution It is easy to show that
\begin{equation}
\delta(n) \system{Z} 1
\end{equation}
and from \eqref{eq:unit_step},
\begin{align}
U(z) &= \sum _{n= 0}^{\infty}z^{-n}
\\
&=\frac{1}{1-z^{-1}}, \quad \abs{z} > 1
\end{align}
using the formula for the sum of an infinite geometric progression.
%
\item Show that 
\begin{equation}
\label{eq:anun}
a^nu(n) \system{Z} \frac{1}{1-az^{-1}} \quad \abs{z} > \abs{a}
\end{equation}
\solution 
From $\eqref{eq:z_trans}$
\begin{align}
	a^nu(n) &\system{Z} \sum_{n = 0}^{\infty}\brak{az^{-1}}^n \\
			&= \frac{1}{1-az^{-1}}  \quad \forall \abs{z} > \abs{a}
\end{align}
%
\\
\item Let
\begin{equation}
H\brak{e^{\j \omega}} = H\brak{z = e^{\j \omega}}.
\end{equation}
Plot $\abs{H\brak{e^{\j \omega}}}$.  Comment.  $H(e^{\j \omega})$ is
known as the {\em Discrete Time Fourier Transform} (DTFT) of $x(n)$.
\\
\solution The below code plots $|H(e^{j\omega})|$.
\begin{lstlisting}
link
\end{lstlisting}
Substituting $z = e^{j\omega}$ in \eqref{eq:freq_resp}, we get
\begin{align}
	|H\brak{e^{j\omega}}| &= \left|\frac{1 + e^{-2j\omega}}{1 + \frac{1}{2}e^{-j\omega}}\right| \\
									  &= \sqrt{\frac{\brak{1 + \cos{(2\omega)}}^2 + \brak{\sin{(2\omega)}}^2}{\brak{1 + \frac{1}{2}\cos{(\omega)}}^2 + \brak{\frac{1}{2}\sin{(\omega)}}^2}}\\
									  &= \frac{4|\cos{(\omega)}|}{\sqrt{5 + 4\cos{(\omega)}}}
\end{align}
\begin{align}
	\left|H\brak{e^{j\brak{\omega + 2\pi}}}\right| &= \frac{4|\cos\brak{\omega + 2\pi}|}{\sqrt{5 + 4\cos\brak{\omega + 2\pi}}} \\
											   &= \frac{4|\cos{(\omega)}|}{\sqrt{5 + 4\cos{(\omega)}}} \\
											   &= \left|H\brak{e^{(j\omega)}}\right|	
\end{align}
Its fundamental period is $2\pi$ , which verifies that the DTFT of a signal is always periodic.

\begin{figure}[H]
\centering
\includegraphics[width=\columnwidth]{figs/H(z)_3.5.png}
\caption{$\abs{H\brak{e^{j\omega}}}$}
\label{fig:H(z)_3.5}
\end{figure}
\end{enumerate}

\section{Impulse Response}
\begin{enumerate}[label=\thesection.\arabic*]
\item \label{prob:impulse_resp}
Find an expression for $h(n)$ using $H(z)$, given that 
%in Problem \ref{eq:ztransab} and \eqref{eq:anun}, given that
\begin{equation}
\label{eq:impulse_resp}
h(n) \system{Z} H(z)
\end{equation}
and there is a one to one relationship between $h(n)$ and $H(z)$. $h(n)$ is known as the {\em impulse response} of the
system defined by \eqref{prob:2.2}.
\\
\solution From \eqref{eq:freq_resp},
\begin{align}
H(z) &= \frac{1}{1 + \frac{1}{2}z^{-1}} + \frac{ z^{-2}}{1 + \frac{1}{2}z^{-1}}, 
\quad \abs{z} > \frac{1}{2} \label{eq:H(z)}\\
\implies h(n) &= \brak{-\frac{1}{2}}^{n}(u(n)) + \brak{-\frac{1}{2}}^{n-2}(u(n-2))
\end{align}
using \eqref{eq:anun} and \eqref{eq:z_trans_shift}.
\item Sketch $h(n)$. Is it bounded? Convergent? 
\\
\solution The following code plots $h(n)$ 
\begin{lstlisting}
link
\end{lstlisting}
\begin{figure}[H]
\centering
\includegraphics[width=\columnwidth]{figs/hn}
\caption{$h(n)$ }
\label{fig:hn}
\end{figure}
%
\item The system with $h(n)$ is defined to be stable if
\begin{equation}
\sum_{n=-\infty}^{\infty}h(n) < \infty \label{eq:stabilty_condn}
\end{equation}
Is the system defined by \eqref{prob:2.2} stable for the impulse response in \eqref{eq:impulse_resp}?\\
\solution For a stable system, the R.O.C of $H(z)$ should contain the boundary of unit circle $|z|=1$.From $\eqref{eq:H(z)}$\\
\begin{align}
    \abs{z} =1 \subset \abs{z} > \frac{1}{2}
\end{align}
Therefore it converges and hence stable.\\
\item 
Compute and sketch $h(n)$ using 
\begin{equation}
\label{eq:iir_filter_h}
h(n) + \frac{1}{2}h(n-1) = \delta(n) + \delta(n-2), 
\end{equation}
%
This is the definition of $h(n)$.
\\
\solution\\
Below is the plot of $h(n)$ using the difference equations.
\ref{fig:hn}. 

\begin{lstlisting}
link 4.2.py
\end{lstlisting}
\begin{figure}[!ht]
\centering
\includegraphics[width=\columnwidth]{figs/hndef}
\caption{$h(n)$ from definition is same as  \figref{fig:hn}}
\label{fig:hndef}
\end{figure}
%
\item Compute 
%
\begin{equation}
\label{eq:convolution}
y(n) = x(n)*h(n) = \sum_{n=-\infty}^{\infty}x(k)h(n-k)
\end{equation}
%
Comment. The operation in \eqref{eq:convolution} is known as
{\em convolution}.
%
\\
\solution The following code plots Fig. \ref{fig:ynconv} using convolution.
\ref{fig:xnyn}. 
%
\begin{lstlisting}
link
\end{lstlisting}
\begin{figure}[H]
\centering
\includegraphics[width=\columnwidth]{figs/y_by_conv.png}
\caption{$y(n)$ from the definition of convolution}
\label{fig:ynconv}
\end{figure}

\item Show that
\begin{equation}
y(n) =  \sum_{n=-\infty}^{\infty}x(n-k)h(k)
\end{equation}
\solution
In \eqref{eq:convolution},substituting $k \rightarrow n - k$, we get
\begin{align}
y\brak{n} &= \sum_{n - k=-\infty}^{\infty}x\brak{n - k}h\brak{k} \\
		  &= \sum_{k=-\infty}^{\infty}x\brak{n - k}h\brak{k}
\end{align}
\end{enumerate}



\section{DFT and FFT}
\begin{enumerate}[label=\thesection.\arabic*]
\item
Compute
\begin{equation}
X(k) \define \sum _{n=0}^{N-1}x(n) e^{-\j2\pi kn/N}, \quad k = 0,1,\dots, N-1
\end{equation}
and $H(k)$ using $h(n)$.
\item Compute 
\begin{equation}
Y(k) = X(k)H(k)
\label{eq:fp}
\end{equation}
\item Compute
\begin{equation}
y\brak{n}={\frac {1}{N}}\sum _{k=0}^{N-1}Y\brak{k}\cdot e^{\j 2\pi kn/N},\quad n = 0,1,\dots, N-1
\label{eq:inv-ft}
\end{equation}

\solution The above three questions are solved using the following code :
\begin{lstlisting}
link 5_sol.py
\end{lstlisting}

\item Repeat the previous exercise by computing $X(k), H(k)$ and $y(n)$ through FFT and IFFT.
\solution The solution of this question can be found in the code below.
\begin{lstlisting}
link 5.4.py   
\end{lstlisting}
This code verifies the result by plotting the obtained result with the result obtained previously.
\begin{figure}[H]
\centering
\includegraphics[width=\columnwidth]{figs/yn_verf_5.4.png}
\caption{$y(n)$ obtained from IDFT and IFFT is plotted and verified}
\label{fig:yn_verf_5.4}
\end{figure}

\item Wherever possible, express all the above equations as matrix equations.\\
\solution The DFT matrix is defined as : 
\begin{align}
	\mtx{W} = 
	\begin{pmatrix}
		\omega^0 & \omega^0 & \ldots & \omega^0 \\
		\omega^0 & \omega^1 & \ldots & \omega^{N - 1} \\
		\vdots & \vdots & \ddots & \vdots \\
		\omega^0 & \omega^{N - 1} & \ldots & \omega^{(N -1)(N - 1)}
	\end{pmatrix}
\end{align}
where $\omega=e^{-\frac{j2\pi}{N}}$ . Now any DFT equation can be written as
\begin{align}
    \mtx{X} = \mtx{W}\mtx{x}
\end{align}
\noindent where
\begin{align}
	\mtx{x} = 
	\begin{pmatrix}
		x(0) \\ x(1) \\ \vdots \\ x(n - 1)
	\end{pmatrix}
\end{align}
\begin{align}
	\mtx{X} = 
	\begin{pmatrix}
		X(0) \\ X(1) \\ \vdots \\ X(n - 1)
	\end{pmatrix}
\end{align}
Thus we can rewrite  \eqref{eq:fp} as:
\begin{align}
	\mtx{Y} = \mtx{X}\odot\mtx{H} = \brak{\mtx{W}\mtx{x}}\odot\brak{\mtx{W}\mtx{h}}
\end{align}
where the $\odot$ represents the Hadamard product which performs element-wise multiplication.
\end{enumerate}

The below code computes $y\brak{n}$ by DFT Matrix and then plots it.
\begin{lstlisting}
link 5.5.py
\end{lstlisting}
\begin{figure}[H]
\centering
\includegraphics[width=\columnwidth]{figs/yn_DFT_matrix.png}
\caption{$y(n)$ obtained from DFT through matrix computations}
\label{fig:yn_DFT_matrix}
\end{figure}



\section{\underline{EXERCISES}}
\noindent Answer the following questions by looking at the python code in Problem \ref{prob:audio_filter_code}.


\begin{enumerate}[label=\thesection.\arabic*]
\item
The command
\begin{lstlisting}
	output_signal = signal.lfilter(b, a, input_signal)
	\end{lstlisting}
in Problem \ref{prob:audio_filter_code} is executed through the following difference equation
\begin{equation}
\label{eq:iir_filter_gen}
 \sum _{m=0}^{M}a\brak{m}y\brak{n-m}=\sum _{k=0}^{N}b\brak{k}x\brak{n-k} 
\end{equation}
%
where the input signal is $x(n)$ and the output signal is $y(n)$ with initial values all 0. Replace
\textbf{signal. filtfilt} with your own routine and verify.\\

\solution The below code gives the output of an Audio Filter without using the built in function signal.lfilter(Slightly modified audio file here).
\begin{lstlisting}
link 6.1.py 
\end{lstlisting}
\begin{figure}[H]
\centering
\includegraphics[width=\columnwidth]{figs/Audio_Filter_verf.png}
\caption{Both the outputs using and without using function overlap}
\label{fig:6.1}
\end{figure}





\item Repeat all the exercises in the previous sections for the above $a$ and $b$.\\
\solution The code in \ref{prob:audio_filter_code} generates the values of $a$ and $b$  which can be used to generate a difference equation.\\
And,
\begin{align}
    M &= 5\\
    N&=5
\end{align}
From \ref{eq:iir_filter_gen} 
\begin{align}
    &a\brak{0}y\brak{n} + a\brak{1}y\brak{n-1}+a\brak{2}y\brak{n-2}+a\brak{3}\\ \notag &y\brak{n-3} + a\brak{4}y\brak{n-4} =   b\brak{0}x\brak{n} + b\brak{1}x\brak{n-1}\\ \notag &+b\brak{2}x\brak{n-2}+b\brak{3}x\brak{n-3} + b\brak{4}x\brak{n-4} 
\end{align}
Difference Equation is given by :
\begin{align}
	&y(n) - \brak{3.658}y(n - 1) + \brak{5.0314}y(n - 2) \nonumber \\
	&- \brak{3.083}y(n - 3) + \brak{0.710}y(n - 4) \nonumber \\
	&= \brak{1.555\times 10^{-5}}x(n) + \brak{6.220 \times 10^{-5}}x(n - 1) \nonumber \\
	&+ \brak{9.331 \times 10^{-5}}x(n - 2) + \brak{6.220 \times 10^{-5}}x(n - 3) \nonumber \\
	&+ \brak{1.555 \times 10^{-5}}x(n - 4)
\end{align}
From \eqref{eq:iir_filter_gen} 
\begin{align}
    H(z) &= \frac{b_0 + b_1 z^{-1} + b_2 z^{-2} + \ldots + b_M z^{-N}}{a_0 + a_1 z^{-1} + a_2 z^{-2} + \ldots + a_N z^{-M}}\\
    H(z) &= \frac{\sum_{k = 0}^{N}b(k)z^{-k}}{\sum_{k = 0}^{M}a(k)z^{-k}} \label{eq:trans-func}
\end{align}
Partial fraction on \eqref{eq:trans-func} can be generalised as:
\begin{align}
    H\brak{z}&= \sum_{i}\frac{r(i)}{1 - p(i)z^{-1}} + \sum_{j}k(j)z^{-j}
	\label{eq:trans-func-pfe}
\end{align}
Now,
\begin{align}
    a^{n}u\brak{n} \system{Z} \frac{1}{1-az^{-1}} \label{eq:res-1}\\
    \delta\brak{n-k} \system{Z} z^{-k}\label{eq:res-2}
\end{align}
Taking inverse z transform of \eqref{eq:trans-func-pfe} by using \eqref{eq:res-1} and \eqref{eq:res-2}
\begin{align}
h(n) &= \sum_{i}r(i)[p(i)]^nu(n) + \sum_{j}k(j)\delta(n - j)
	\label{eq:h-n-expr}
\end{align}
The below code computes the values of $r\brak{i},p\brak{i} , k\brak{i}$ and plots $h\brak{n}$
\begin{lstlisting}
link 6.2.py
\end{lstlisting}
\input{tables/audio_filter_tab}

\textbf{Stability of h(n)}:\\
According to \eqref{eq:stabilty_condn}
\begin{align}
H\brak{z} &= \sum_{n = 0}^{\infty} h\brak{n}z^{-n}\\
H(1)&= \sum_{n = 0}^{\infty}h(n)  = \frac{\sum_{k = 0}^{N}b(k)}{\sum_{k = 0}^{M}a(k)}< \infty
\end{align}
As both $a\brak{k}$ and $b\brak{k}$ are finite length sequences they converge.\\
The below code plots Filter frequency response
\begin{lstlisting}
link 6_filter_response.py
\end{lstlisting}
\begin{figure}[H]
\centering
\includegraphics[width=1\columnwidth]{figs/Filter_Response.png}
\caption{Frequency Response of Audio Filter}
\label{fig:H(w)_6}
\end{figure}
The below code plots the Butterworth Filter in analog domain by using bilinear transform.
\begin{align}
    z=\frac{1+sT/2}{1-sT/2}
\end{align}
\begin{lstlisting}
link analog_filt.py
\end{lstlisting}
\begin{figure}[H]
\centering
\includegraphics[width=1\columnwidth]{figs/Butterworth_analog.png}
\caption{Butterworth Filter Frequency response in analog domain}
\label{fig:H(w)_6}
\end{figure}



The below code plots the Pole-Zero Plot of the frequency response.
\begin{lstlisting}
link pole zero
\end{lstlisting}
\begin{figure}[H]
\centering
\includegraphics[width=1\columnwidth]{figs/Pole_Zero_Plt.png}
\caption{There are complex poles. So $h\brak{n}$ should be damped sinusoid.}
\label{fig:pole_zero_6.2}
\end{figure}

\begin{figure}[H]
\centering
\includegraphics[width=\columnwidth]{figs/h(n)_6.2.png}
\caption{h(n) of Audio Filter.It is a damped sinusoid.}
\label{fig:6.2_hn}
\end{figure}

\item What is the sampling frequency of the input signal?\\
\solution The Sampling Frequency is 48KHz
\item
What is type, order and  cutoff-frequency of the above butterworth filter
\\
\solution The given butterworth filter is lowpass with order=4 and cutoff-frequency=1kHz.

\item
Modify the code with different input parameters and get the best possible output.

\solution
A better filtering was found on setting the order of the filter to be 5.

\end{enumerate}


\end{document}

